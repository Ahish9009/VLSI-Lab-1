
\documentclass[a4paper,11pt]{report}
\begin{document}
\title{VLSI Assignment - 1 Report}
\author{Ahish Deshpande}
\date{\today}
\maketitle


\tableofcontents
\begin{enumerate}

\item Aim
\item Introduction
\item Procedure
\item Result and Discussion
\item Conclusion
\item References

\end{enumerate}
\newpage

\renewcommand\thesection{\arabic{section}}
\section{Aim}
The aim of this assignment is to write Verilog code and testbench for the following combinational gates:
  \begin{center}\textbf{AND, OR, NOR, NAND, XOR, NOT} and \textbf{XNOR}. \end{center}\\\\
We also want to write Verilog code and test bench for the following compound combinational circuits.
\begin{enumerate}
\item \textbf{Y = AB + CD}
\item \textbf{Y = (ABC + DE).F}
\item \textbf{Y = ((A+B')(CD+E))'}
\end{enumerate} \\
We will also write their truth table and find the:
\begin{itemize}
\item RTL Schematics
\item Area
\item Power
\item Timing Diagram
\end{itemize}

\section{Introduction}
For this assignment, we will be using \textbf{Xilinx ISE}, and \textbf{Cadence}.
\\ We will generate the following using Xilinx ISE:
\begin{itemize}
\item RTL Schematics
\item Timing Diagram
\end{itemize}
\\ We will generate the following using \textbf{Cadence}:
\begin{itemize}
\item Area Report
\item Power Report
\item Timing Report
\item Netlist
\end{itemize}

\section{Procedure}
The following procedure will be followed:
\begin{enumerate}
\item Write the required Verilog code and testbench using \textbf{Xilinx ISE}
\item Using the above, generate the RTL Schematics and Timing Diagram in Xilinx
\item Using the Verilog code already written, use \textbf{Cadence NCLaunch} to generate a Timing Diagram
\item Next, use \textbf{Cadence Genus} to generate Area, Power and Timing reports of the gates and combinational circuits generated. The netlist will also be generated here
\item Use the generated netlist with \textbf{Cadence Innovus} to optimize the design. The Timing reports for pre-placement, post-placement, pre-rout and post-rout are then generated along with the Power and Area reports. The optimized netlist is also generated.
\end{enumerate}

\section{Result \& Discussion}

The attached screenshots show the values for the function for all values of the input(thus generating the required truth table). \\
Tabulating the values observed in the reports from Genus(without optimization) and Innovus(with optimization): \\

\begin{tabular}{|c|c|c|c|c|}
  \hline
   \textbf{Design} & \multicolumn{2}{|c|}{\textbf{Genus}} & \multicolumn{2}{|c|}{\textbf{Innovus}} \\
   \hline
    & \textbf{Area}\textit{(units)} & \textbf{Power}\textit{(nW)} & \textbf{Area}\textit{(units)} & \textbf{Power}\textit{(nW)} \\
    \hline
    AND & 5 & 112.983 &  4.5414  & 90.69  \\
    \hline
    OR & 5 & 135.769 & 4.5414 & 135.2 \\
    \hline
    NOR & 3 & 92.834 & 3.027 & 89.96 \\
    \hline
    NAND & 3 & 69.083 & 3.027 & 49.93\\
    \hline
    XOR & 8 & 340.607 & 8.325 & 235.4\\
    \hline
    NOT & 2 & 53.174 & 2.270 & 48.51\\
    \hline
    XNOR & 8 & 204.259 & 7.569 & 134.3\\
    \hline
    F1 & 8 & 218.452 & 7.569 & 169.8\\
    \hline
    F2 & 11 & 325.562 & 11.353 & 213.6 \\
    \hline
    F3 & 13 & 367.735 & 12.867 & 285.3 \\
    \hline

\end{tabular} \\

\begin{enumerate}
\item We observe that the \textbf{power} dissipated by the design is proportional to the \textbf{area/size} of the design. 
\item We can also observe that the power consumed by the design \textbf{can be significantly reduced} by optimizing the design.

\end{enumerate}

\section{Conclusion}

We can conclude that:
\begin{itemize}
\item The required designs were successfully implemented using \textbf{Xilinx ISE} and \textbf{Cadence}
\item The power dissipated by the circuits is \textbf{proportional} to the size of the design
\item The power consumption of the design can be significantly reduced by \textbf{optimizing the design} appropriately
\end{itemize}

\section{References}
\begin{enumerate}
\item \textit{www.asic-world.com}
\item \textit{www.xilinx.com}
\item \textit{www.cadence.com}
\item \textit{electronics.stackexchange.com}
\item \textit{Tutorials and Lab Sessions}
\end{enumerate}




\end{document}

